% \iffalse meta-comment
%
%% File: l3fp-logic.dtx Copyright (C) 2011-2014 The LaTeX3 Project
%%
%% It may be distributed and/or modified under the conditions of the
%% LaTeX Project Public License (LPPL), either version 1.3c of this
%% license or (at your option) any later version.  The latest version
%% of this license is in the file
%%
%%    http://www.latex-project.org/lppl.txt
%%
%% This file is part of the "l3kernel bundle" (The Work in LPPL)
%% and all files in that bundle must be distributed together.
%%
%% The released version of this bundle is available from CTAN.
%%
%% -----------------------------------------------------------------------
%%
%% The development version of the bundle can be found at
%%
%%    http://www.latex-project.org/svnroot/experimental/trunk/
%%
%% for those people who are interested.
%%
%%%%%%%%%%%
%% NOTE: %%
%%%%%%%%%%%
%%
%%   Snapshots taken from the repository represent work in progress and may
%%   not work or may contain conflicting material!  We therefore ask
%%   people _not_ to put them into distributions, archives, etc. without
%%   prior consultation with the LaTeX Project Team.
%%
%% -----------------------------------------------------------------------
%%
%
%<*driver>
\documentclass[full]{l3doc}
\GetIdInfo$Id: l3fp-logic.dtx 5893 2015-08-26 16:16:52Z mittelba $
  {L3 Floating-point conditionals}
\begin{document}
  \DocInput{\jobname.dtx}
\end{document}
%</driver>
% \fi
%
% \title{The \textsf{l3fp-logic} package\thanks{This file
%         has version number \ExplFileVersion, last
%         revised \ExplFileDate.}\\
% Floating point conditionals}
% \author{^^A
%  The \LaTeX3 Project\thanks
%    {^^A
%      E-mail:
%        \href{mailto:latex-team@latex-project.org}
%          {latex-team@latex-project.org}^^A
%    }^^A
% }
% \date{Released \ExplFileDate}
%
% \maketitle
%
% \begin{documentation}
%
% \end{documentation}
%
% \begin{implementation}
%
% \section{\pkg{l3fp-logic} Implementation}
%
%    \begin{macrocode}
%<*initex|package>
%    \end{macrocode}
%
%    \begin{macrocode}
%<@@=fp>
%    \end{macrocode}
%
% \subsection{Syntax of internal functions}
%
% \begin{itemize}
%   \item \cs{@@_compare_npos:nwnw} \Arg{expo_1} \meta{body_1} |;|
%     \Arg{expo_2} \meta{body_2} |;|
%   \item \cs{@@_minmax_o:Nw} \meta{sign} \meta{floating point array}
%   \item \cs{@@_not_o:w} |?| \meta{floating point array} (with one floating point number only)
%   \item \cs{@@_&_o:ww} \meta{floating point} \meta{floating point}
%   \item \cs{@@_|_o:ww} \meta{floating point} \meta{floating point}
%   \item \cs{@@_ternary:NwwN}, \cs{@@_ternary_auxi:NwwN},
%     \cs{@@_ternary_auxii:NwwN} have to be understood.
% \end{itemize}
%
% \subsection{Existence test}
%
% \begin{macro}[pTF]{\fp_if_exist:N, \fp_if_exist:c}
%   Copies of the \texttt{cs} functions defined in \pkg{l3basics}.
%    \begin{macrocode}
\prg_new_eq_conditional:NNn \fp_if_exist:N \cs_if_exist:N { TF , T , F , p }
\prg_new_eq_conditional:NNn \fp_if_exist:c \cs_if_exist:c { TF , T , F , p }
%    \end{macrocode}
% \end{macro}
%
% \subsection{Comparison}
%
% \begin{macro}[pTF, EXP]{\fp_compare:n}
% \begin{macro}[aux, EXP]{\@@_compare_return:w}
%   Within floating point expressions, comparison operators are treated
%   as operations, so we evaluate |#1|, then compare with $0$.
%    \begin{macrocode}
\prg_new_conditional:Npnn \fp_compare:n #1 { p , T , F , TF }
  {
    \exp_after:wN \@@_compare_return:w
    \exp:w \exp_end_continue_f:w \@@_parse:n {#1}
  }
\cs_new:Npn \@@_compare_return:w \s_@@ \@@_chk:w #1#2;
  {
    \if_meaning:w 0 #1
      \prg_return_false:
    \else:
      \prg_return_true:
    \fi:
  }
%    \end{macrocode}
% \end{macro}
% \end{macro}
%
% \begin{macro}[pTF, EXP]{\fp_compare:nNn}
% \begin{macro}[aux, EXP]{\@@_compare_aux:wn}
%   Evaluate |#1| and |#3|, using an auxiliary to expand both, and feed
%   the two floating point numbers swapped to \cs{@@_compare_back:ww},
%   defined below.  Compare the result with |`#2-`=|, which is $-1$ for
%   |<|, $0$ for |=|, $1$ for |>| and $2$ for |?|.
%    \begin{macrocode}
\prg_new_conditional:Npnn \fp_compare:nNn #1#2#3 { p , T , F , TF }
  {
    \if_int_compare:w
        \exp_after:wN \@@_compare_aux:wn
          \exp:w \exp_end_continue_f:w \@@_parse:n {#1} {#3}
        = \__int_eval:w `#2 - `= \__int_eval_end:
      \prg_return_true:
    \else:
      \prg_return_false:
    \fi:
  }
\cs_new:Npn \@@_compare_aux:wn #1; #2
  {
    \exp_after:wN \@@_compare_back:ww
      \exp:w \exp_end_continue_f:w \@@_parse:n {#2} #1;
  }
%    \end{macrocode}
% \end{macro}
% \end{macro}
%
% \begin{macro}[aux, EXP]{\@@_compare_back:ww, \@@_compare_nan:w}
%   \begin{quote}
%     \cs{@@_compare_back:ww} \meta{y} |;| \meta{x} |;|
%   \end{quote}
%   Expands (in the same way as \cs{int_eval:n}) to $-1$ if $x<y$, $0$
%   if $x=y$, $1$ if $x>y$, and $2$ otherwise (denoted as $x?y$).  If
%   either operand is \texttt{nan}, stop the comparison with
%   \cs{@@_compare_nan:w} returning $2$.  If $x$ is negative, swap the
%   outputs $1$ and $-1$ (\emph{i.e.}, $>$ and $<$); we can henceforth
%   assume that $x\geq 0$.  If $y\geq 0$, and they have the same type,
%   either they are normal and we compare them with
%   \cs{@@_compare_npos:nwnw}, or they are equal.  If $y\geq 0$, but of
%   a different type, the highest type is a larger number.  Finally, if
%   $y\leq 0$, then $x>y$, unless both are zero.
%    \begin{macrocode}
\cs_new:Npn \@@_compare_back:ww
    \s_@@ \@@_chk:w #1 #2 #3;
    \s_@@ \@@_chk:w #4 #5 #6;
  {
    \__int_value:w
      \if_meaning:w 3 #1 \exp_after:wN \@@_compare_nan:w \fi:
      \if_meaning:w 3 #4 \exp_after:wN \@@_compare_nan:w \fi:
      \if_meaning:w 2 #5 - \fi:
      \if_meaning:w #2 #5
        \if_meaning:w #1 #4
          \if_meaning:w 1 #1
            \@@_compare_npos:nwnw #6; #3;
          \else:
            0
          \fi:
        \else:
          \if_int_compare:w #4 < #1 - \fi: 1
        \fi:
      \else:
        \if_int_compare:w #1#4 = \c_zero
          0
        \else:
          1
        \fi:
      \fi:
    \exp_stop_f:
  }
\cs_new:Npn \@@_compare_nan:w #1 \exp_stop_f: { \c_two }
%    \end{macrocode}
% \end{macro}
%
% \begin{macro}[int, EXP]{\@@_compare_npos:nwnw}
% \begin{macro}[aux, EXP]{\@@_compare_significand:nnnnnnnn}
%   \begin{quote}
%     \cs{@@_compare_npos:nwnw}
%       \Arg{expo_1} \meta{body_1} |;|
%       \Arg{expo_2} \meta{body_2} |;|
%   \end{quote}
%   Within an \cs{__int_value:w} \ldots{} \cs{exp_stop_f:} construction,
%   this expands to $0$ if the two numbers are equal, $-1$ if the first
%   is smaller, and $1$ if the first is bigger.  First compare the
%   exponents: the larger one denotes the larger number.  If they are
%   equal, we must compare significands.  If both the first $8$ digits and
%   the next $8$ digits coincide, the numbers are equal.  If only the
%   first $8$ digits coincide, the next $8$ decide.  Otherwise, the
%   first $8$ digits are compared.
%    \begin{macrocode}
\cs_new:Npn \@@_compare_npos:nwnw #1#2; #3#4;
  {
    \if_int_compare:w #1 = #3 \exp_stop_f:
      \@@_compare_significand:nnnnnnnn #2 #4
    \else:
      \if_int_compare:w #1 < #3 - \fi: 1
    \fi:
  }
\cs_new:Npn \@@_compare_significand:nnnnnnnn #1#2#3#4#5#6#7#8
  {
    \if_int_compare:w #1#2 = #5#6 \exp_stop_f:
      \if_int_compare:w #3#4 = #7#8 \exp_stop_f:
        0
      \else:
        \if_int_compare:w #3#4 < #7#8 - \fi: 1
      \fi:
    \else:
      \if_int_compare:w #1#2 < #5#6 - \fi: 1
    \fi:
  }
%    \end{macrocode}
% \end{macro}
% \end{macro}
%
% \subsection{Floating point expression loops}
%
% \begin{macro}[rEXP]
%   {
%     \fp_do_until:nn, \fp_do_while:nn,
%     \fp_until_do:nn, \fp_while_do:nn
%   }
%   These are quite easy given the above functions. The |do_until| and
%   |do_while| versions execute the body, then test.  The |until_do| and
%   |while_do| do it the other way round.
%    \begin{macrocode}
\cs_new:Npn \fp_do_until:nn #1#2
  {
    #2
    \fp_compare:nF {#1}
      { \fp_do_until:nn {#1} {#2} }
  }
\cs_new:Npn \fp_do_while:nn #1#2
  {
    #2
    \fp_compare:nT {#1}
      { \fp_do_while:nn {#1} {#2} }
  }
\cs_new:Npn \fp_until_do:nn #1#2
  {
    \fp_compare:nF {#1}
      {
        #2
        \fp_until_do:nn {#1} {#2}
      }
  }
\cs_new:Npn \fp_while_do:nn #1#2
  {
    \fp_compare:nT {#1}
      {
        #2
        \fp_while_do:nn {#1} {#2}
      }
  }
%    \end{macrocode}
% \end{macro}
%
% \begin{macro}[rEXP]
%   {
%     \fp_do_until:nNnn, \fp_do_while:nNnn,
%     \fp_until_do:nNnn, \fp_while_do:nNnn
%   }
%   As above but not using the |nNn| syntax.
%    \begin{macrocode}
\cs_new:Npn \fp_do_until:nNnn #1#2#3#4
  {
    #4
    \fp_compare:nNnF {#1} #2 {#3}
      { \fp_do_until:nNnn {#1} #2 {#3} {#4} }
  }
\cs_new:Npn \fp_do_while:nNnn #1#2#3#4
  {
    #4
    \fp_compare:nNnT {#1} #2 {#3}
      { \fp_do_while:nNnn {#1} #2 {#3} {#4} }
  }
\cs_new:Npn \fp_until_do:nNnn #1#2#3#4
  {
    \fp_compare:nNnF {#1} #2 {#3}
      {
        #4
        \fp_until_do:nNnn {#1} #2 {#3} {#4}
      }
  }
\cs_new:Npn \fp_while_do:nNnn #1#2#3#4
  {
    \fp_compare:nNnT {#1} #2 {#3}
      {
        #4
        \fp_while_do:nNnn {#1} #2 {#3} {#4}
      }
  }
%    \end{macrocode}
% \end{macro}
%
% \subsection{Extrema}
%
% \begin{macro}[int, EXP]{\@@_minmax_o:Nw}
%   The argument~|#1| is $2$~to find the maximum of an array~|#2| of
%   floating point numbers, and $0$~to find the minimum.  We read
%   numbers sequentially, keeping track of the largest (smallest) number
%   found so far.  If numbers are equal (for instance~$\pm0$), the first
%   is kept.  We append $-\infty$ ($\infty$), for the case of an empty
%   array, currently impossible.  Since no number is smaller (larger)
%   than that, it will never alter the maximum (minimum).  The weird
%   fp-like trailing marker breaks the loop correctly: see the precise
%   definition of \cs{@@_minmax_loop:Nww}.
%    \begin{macrocode}
\cs_new:Npn \@@_minmax_o:Nw #1#2 @
  {
    \if_meaning:w 0 #1
      \exp_after:wN \@@_minmax_loop:Nww \exp_after:wN \c_one
    \else:
      \exp_after:wN \@@_minmax_loop:Nww \exp_after:wN \c_minus_one
    \fi:
    #2
    \s_@@ \@@_chk:w 2 #1 \s_@@_exact ;
    \s_@@ \@@_chk:w { 3 \@@_minmax_break_o:w } ;
  }
%    \end{macrocode}
% \end{macro}
%
% \begin{macro}[aux, EXP]{\@@_minmax_loop:Nww}
%   The first argument is $-1$ or $1$ to denote the case where the
%   currently largest (smallest) number found (first floating point
%   argument) should be replaced by the new number (second floating
%   point argument).  If the new number is \texttt{nan}, keep that as
%   the extremum, unless that extremum is already a \texttt{nan}.
%   Otherwise, compare the two numbers.  If the new number is larger (in
%   the case of |max|) or smaller (in the case of |min|), the test
%   yields \texttt{true}, and we keep the second number as a new
%   maximum; otherwise we keep the first number.  Then loop.
%    \begin{macrocode}
\cs_new:Npn \@@_minmax_loop:Nww
    #1 \s_@@ \@@_chk:w #2#3; \s_@@ \@@_chk:w #4#5;
  {
    \if_meaning:w 3 #4
      \if_meaning:w 3 #2
        \@@_minmax_auxi:ww
      \else:
        \@@_minmax_auxii:ww
      \fi:
    \else:
      \if_int_compare:w
          \@@_compare_back:ww
            \s_@@ \@@_chk:w #4#5;
            \s_@@ \@@_chk:w #2#3;
          = #1
        \@@_minmax_auxii:ww
      \else:
        \@@_minmax_auxi:ww
      \fi:
    \fi:
    \@@_minmax_loop:Nww #1
      \s_@@ \@@_chk:w #2#3;
      \s_@@ \@@_chk:w #4#5;
  }
%    \end{macrocode}
% \end{macro}
%
% \begin{macro}[aux, EXP]{\@@_minmax_auxi:ww, \@@_minmax_auxii:ww}
%   Keep the first/second number, and remove the other.
%    \begin{macrocode}
\cs_new:Npn \@@_minmax_auxi:ww  #1 \fi: \fi: #2 \s_@@ #3 ; \s_@@ #4;
  { \fi: \fi: #2 \s_@@ #3 ; }
\cs_new:Npn \@@_minmax_auxii:ww #1 \fi: \fi: #2 \s_@@ #3 ;
  { \fi: \fi: #2 }
%    \end{macrocode}
% \end{macro}
%
% \begin{macro}[aux, EXP]{\@@_minmax_break_o:w}
%   This function is called from within an \cs{if_meaning:w} test.  Skip
%   to the end of the tests, close the current test with \cs{fi:}, clean
%   up, and return the appropriate number with one post-expansion.
%    \begin{macrocode}
\cs_new:Npn \@@_minmax_break_o:w #1 \fi: \fi: #2 \s_@@ #3; #4;
  { \fi: \@@_exp_after_o:w \s_@@ #3; }
%    \end{macrocode}
% \end{macro}
%
% \subsection{Boolean operations}
%
% \begin{macro}[int, EXP]{\@@_not_o:w}
%   Return \texttt{true} or \texttt{false}, with two expansions, one to
%   exit the conditional, and one to please \pkg{l3fp-parse}.  The first
%   argument is provided by \pkg{l3fp-parse} and is ignored.
%    \begin{macrocode}
\cs_new:cpn { @@_not_o:w } #1 \s_@@ \@@_chk:w #2#3; @
  {
    \if_meaning:w 0 #2
      \exp_after:wN \exp_after:wN \exp_after:wN \c_one_fp
    \else:
      \exp_after:wN \exp_after:wN \exp_after:wN \c_zero_fp
    \fi:
  }
%    \end{macrocode}
% \end{macro}
%
% \begin{macro}[int, EXP]+\@@_&_o:ww+
% \begin{macro}[int, EXP]+\@@_|_o:ww+
% \begin{macro}[aux, EXP]{\@@_and_return:wNw}
%   For \texttt{and}, if the first number is zero, return it (with the
%   same sign).  Otherwise, return the second one.  For \texttt{or}, the
%   logic is reversed: if the first number is non-zero, return it,
%   otherwise return the second number: we achieve that by hi-jacking
%   \cs{@@_\&_o:ww}, inserting an extra argument, \cs{else:}, before
%   \cs{s_@@}.  In all cases, expand after the floating point number.
%    \begin{macrocode}
\group_begin:
  \char_set_catcode_letter:N &
  \char_set_catcode_letter:N |
  \cs_new:Npn \@@_&_o:ww #1 \s_@@ \@@_chk:w #2#3;
    {
      \if_meaning:w 0 #2 #1
        \@@_and_return:wNw \s_@@ \@@_chk:w #2#3;
      \fi:
      \@@_exp_after_o:w
    }
  \cs_new_nopar:Npn \@@_|_o:ww { \@@_&_o:ww \else: }
\group_end:
\cs_new:Npn \@@_and_return:wNw #1; \fi: #2#3; { \fi: #2 #1; }
%    \end{macrocode}
% \end{macro}
% \end{macro}
% \end{macro}
%
% \subsection{Ternary operator}
%
%^^A todo: understand and optimize.
% \begin{macro}[int, EXP]
%   {\@@_ternary:NwwN, \@@_ternary_auxi:NwwN, \@@_ternary_auxii:NwwN}
% \begin{macro}[aux, EXP]
%   {
%     \@@_ternary_loop_break:w, \@@_ternary_loop:Nw,
%     \@@_ternary_map_break:, \@@_ternary_break_point:n
%   }
%   The first function receives the test and the true branch of the |?:|
%   ternary operator.  It returns the true branch, unless the test
%   branch is zero.  In that case, the function returns a very specific
%   \texttt{nan}.  The second function receives the output of the first
%   function, and the false branch.  It returns the previous input,
%   unless that is the special \texttt{nan}, in which case we return the
%   false branch.
%    \begin{macrocode}
\cs_new:Npn \@@_ternary:NwwN #1 #2@ #3@ #4
  {
    \if_meaning:w \@@_parse_infix_::N #4
      \@@_ternary_loop:Nw
        #2
        \s_@@ \@@_chk:w { \@@_ternary_loop_break:w } ;
      \@@_ternary_break_point:n { \exp_after:wN \@@_ternary_auxi:NwwN }
      \exp_after:wN #1
      \exp:w \exp_end_continue_f:w
      \@@_exp_after_array_f:w #3 \s_@@_stop
      \exp_after:wN @
      \exp:w
        \@@_parse_operand:Nw \c_two
        \@@_parse_expand:w
    \else:
      \__msg_kernel_expandable_error:nnnn
        { kernel } { fp-missing } { : } { ~for~?: }
      \exp_after:wN \@@_parse_continue:NwN
      \exp_after:wN #1
      \exp:w \exp_end_continue_f:w
      \@@_exp_after_array_f:w #3 \s_@@_stop
      \exp_after:wN #4
      \exp_after:wN #1
    \fi:
  }
\cs_new:Npn \@@_ternary_loop_break:w
    #1 \fi: #2 \@@_ternary_break_point:n #3
  {
    \c_zero = \c_zero \fi:
    \exp_after:wN \@@_ternary_auxii:NwwN
  }
\cs_new:Npn \@@_ternary_loop:Nw \s_@@ \@@_chk:w #1#2;
  {
    \if_int_compare:w #1 > \c_zero
      \exp_after:wN \@@_ternary_map_break:
    \fi:
    \@@_ternary_loop:Nw
  }
\cs_new:Npn \@@_ternary_map_break: #1 \@@_ternary_break_point:n #2 {#2}
\cs_new:Npn \@@_ternary_auxi:NwwN #1#2@#3@#4
  {
    \exp_after:wN \@@_parse_continue:NwN
    \exp_after:wN #1
    \exp:w \exp_end_continue_f:w
    \@@_exp_after_array_f:w #2 \s_@@_stop
    #4 #1
  }
\cs_new:Npn \@@_ternary_auxii:NwwN #1#2@#3@#4
  {
    \exp_after:wN \@@_parse_continue:NwN
    \exp_after:wN #1
    \exp:w \exp_end_continue_f:w
    \@@_exp_after_array_f:w #3 \s_@@_stop
    #4 #1
  }
%    \end{macrocode}
% \end{macro}
% \end{macro}
%
%    \begin{macrocode}
%</initex|package>
%    \end{macrocode}
%
% \end{implementation}
%
% \PrintChanges
%
% \PrintIndex
